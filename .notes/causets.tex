\documentclass[12pt]{article}
\usepackage[top=1in, bottom=1in, left=1in, right=1in]{geometry}
\usepackage{mathtools}
\usepackage{amsmath}
\usepackage{amsfonts}
\usepackage{amssymb}
\DeclareMathSymbol{\comma}{\mathpunct}{letters}{"3B}
\begin{document}
\title{Causal Set Research Notes}
\author{Will Cunningham}
\maketitle

\section{Patch Volume}
\subsection{1+1 Dimensions}

\begin{equation}
\begin{split}
\mathrm{d}V &= (a\mathrm{sec}\eta)^{\mathrm{d}+1} \, \mathrm{d}\eta \, \mathrm{d}\Phi_d \\
   &= a^2\mathrm{sec}^2\eta \, \mathrm{d}\eta \, \mathrm{d}\theta
\end{split}
\end{equation}

\begin{equation}
\begin{split}
V &= a^2\int_0^{\eta_0} \! \mathrm{d}\eta \int_0^{2\pi} \! \mathrm{d}\theta \, \mathrm{sec}^2\eta \\
  &= 2\pi a^2 \int_0^{\eta_0} \! \mathrm{sec}^2\eta \, \mathrm{d}\eta \\
  &= 2\pi a^2\mathrm{tan}\eta_0
\end{split}
\end{equation}

\begin{equation}
\begin{split}
N &= \delta V \\
\therefore N_{\mathrm{1+1}} &= 2\pi\delta a^2 \mathrm{tan}\eta_0
\end{split}
\end{equation}

\subsection{3+1 Dimensions}

\begin{equation}
\mathrm{d}V = a^4 \mathrm{sec}^4\eta \, \mathrm{d}\eta \, \mathrm{sin}^2\phi \, \mathrm{sin}\chi \, \mathrm{d}\phi \, \mathrm{d}\chi \, \mathrm{d}\theta
\end{equation}

\begin{equation}
\begin{split}
V &= a^4 \int_0^{\eta_0} \! \mathrm{d}\eta \int_0^{2\pi} \! \mathrm{d}\theta \int_0^\pi \! \mathrm{d}\chi \int_0^\pi \! \mathrm{d}\phi \, \mathrm{sec}^4\eta \, \mathrm{sin}^2\phi \, \mathrm{sin}\chi \\
  &= 2\pi a^4 \int_0^{\eta_0} \! \mathrm{d}\eta \int_0^\pi \! \mathrm{d}\chi \int_0^\pi \! \mathrm{d}\phi \, \mathrm{sec}^4\eta \, \mathrm{sin}^2\phi \, \mathrm{sin}\chi \\
  &= 4\pi a^4 \int_0^{\eta_0} \! \mathrm{d}\eta \int_0^\pi \! \mathrm{d}\phi \, \mathrm{sec}^4\eta \, \mathrm{sin}^2\phi \\
  &= 2\pi^2 a^4 \int_0^{\eta_0} \! \mathrm{sec}^4\eta \, \mathrm{d}\eta \\
  &= \frac{2}{3}\pi^2 a^4 \, (2+\mathrm{sec}^2\eta_0) \, \mathrm{tan}\eta_0
\end{split}
\end{equation}

\begin{equation}
\begin{split}
N &= \delta V \\
\therefore N_{\mathrm{3+1}} &= \frac{2}{3}\pi^2\delta a^4 \, (2+\mathrm{sec}^2\eta_0) \, \mathrm{tan}\eta_0
\end{split}
\end{equation}

\section{Expected Average Degrees}
\subsection{1+1 Dimensions}

\begin{equation}
\rho = \frac{\mathrm{sec}^2\eta}{\mathrm{tan}\eta_0}
\end{equation}

\begin{equation}
V_p = \mathrm{ln} \, \mathrm{sec}\eta
\end{equation}

\begin{equation}
\begin{split}
\langle V_p \rangle &= \int_0^{\eta_0} \! \frac{\mathrm{sec}^2\eta}{\mathrm{tan}\eta_0} \, \mathrm{ln} \, \mathrm{sec}\eta \, \mathrm{d}\eta \\
  &= \frac{1}{\mathrm{tan}\eta_0} \, \lbrack \eta + \mathrm{tan}\eta (\mathrm{ln} \, \mathrm{sec}\eta - 1) \rbrack_0^{\eta_0} \\
  &= \frac{1}{\mathrm{tan}\eta_0} \, \lbrack \eta_0 + \mathrm{tan}\eta_0 (\mathrm{ln} \, \mathrm{sec}\eta_0 - 1) \rbrack \\
  &= \frac{\eta_0}{\mathrm{tan}\eta_0} + \mathrm{ln} \, \mathrm{sec}\eta_0 - 1
\end{split}
\end{equation}

\begin{equation}
V_f = (\eta_0 - \eta) \, \mathrm{tan}\eta_0 + \mathrm{ln} \, \left(\frac{\mathrm{sec}\eta}{\mathrm{sec}\eta_0}\right)
\end{equation}

\begin{equation}
\begin{split}
\langle V_f \rangle &= \int_0^{\eta_0} \! \frac{\mathrm{sec}^2\eta}{\mathrm{tan}\eta_0} \left[ (\eta_0 - \eta) \, \mathrm{tan}\eta_0 + \ln\left(\frac{\sec\eta}{\sec\eta_0}\right) \right] \, \mathrm{d}\eta \\
  &= \frac{1}{\tan\eta_0} \left[ \eta_0\tan\eta_0 \int_0^{\eta_0} \! \sec^2\eta \, \mathrm{d}\eta - \tan\eta_0 \int_0^{\eta_0} \! \eta\sec^2\eta \, \mathrm{d}\eta \right. \\ & \left.\,\,\,\,\, + \int_0^{\eta_0} \! \sec^2\eta\ln\sec\eta \, \mathrm{d}\eta - \ln\sec\eta_0 \int_0^{\eta_0} \! \sec^2\eta \, \mathrm{d}\eta \right] \\
  &= \frac{1}{\tan\eta_0} \left[ \eta_0\tan^2\eta_0 - \eta_0\tan^2\eta_0 + \tan\eta_0\ln\sec\eta_0 \right. \\ & \left. \,\,\,\,\, + \eta_0 + \tan\eta_0\left(\ln\sec\eta_0 - 1\right) - \tan\eta_0\ln\sec\eta_0 \right] \\
  &= \frac{\eta_0}{\tan\eta_0} + \ln\sec\eta_0 - 1
\end{split}
\end{equation}

\begin{equation}
\begin{split}
\therefore \langle V_p \rangle &= \langle V_f \rangle \\
\therefore \langle \bar{k} \rangle &= 2a^2\delta \left( \langle V_p \rangle + \langle V_f \rangle \right) \\
  &= 4 \delta a^2 \left( \frac{\eta_0}{\tan\eta_0} + \ln\sec\eta_0 - 1 \right)
\end{split}
\end{equation}

\subsection{3+1 Dimensions}

\begin{align}
V_p\left(\eta\right) &= \int_0^\eta \! \mathrm d\eta^\prime \, \int_0^{\eta-\eta^\prime} \! \mathrm d\phi \, \int_0^{2\pi}\!\mathrm d\theta\, \int_0^\pi\!\mathrm d\chi\, a^4\sec^4\eta^\prime\sin^2\phi\sin\chi \\
  &= 2a^4\int_0^\eta\!\mathrm d\eta^\prime\, \int_0^{\eta-\eta^\prime}\!\mathrm d\phi\, \int_0^{2\pi}\!\mathrm d\theta\, \sec^4\eta^\prime\sin^2\phi \\
  &= 4\pi a^4 \int_0^\eta\!\mathrm d\eta^\prime\,\int_0^{\eta-\eta^\prime}\!\mathrm d\phi\, \sec^4\eta^\prime\sin^2\phi \\
  &\qquad\begin{aligned}
    \int_0^{\eta-\eta^\prime}\!\sin^2\phi\,\mathrm d\phi = \frac{1}{2}\left[\eta-\eta^\prime\right.&\left.+\sin\eta\cos\eta\left(1-2\cos^2\eta^\prime\right)\right. \\
    &\left.-\sin\eta^\prime\cos\eta^\prime\left(1-2\cos^2\eta\right)\right]
  \end{aligned} \\
  &\begin{aligned}
    = 2\pi a^4\int_0^\eta\!\mathrm d\eta^\prime\,\sec^4\eta^\prime\left[\eta-\eta^\prime\right.&\left.+\left(1-2\cos^2\eta^\prime\right)\sin\eta\cos\eta\right. \\
    &\left.-\sin\eta^\prime\cos\eta^\prime\left(1-2\cos^2\eta\right)\right]
  \end{aligned} \\
  &\qquad\int_0^\eta\!\eta\sec^4\eta^\prime\,\mathrm d\eta^\prime = \frac{\eta}{3}\tan\eta\left(2+\sec^2\eta\right) \\
  &\qquad\begin{aligned}
    \int_0^\eta\!\eta^\prime\sec^4\eta^\prime\,\mathrm d\eta^\prime = \frac{1}{3}\left[\right.&\left.-\ln\sec^2\eta-\frac{1}{2}\sec^2\eta + \frac{1}{2}\right. \\
    &\left.+2\eta\tan\eta+\eta\sec^2\eta\tan\eta\right]
  \end{aligned} \\
  &\qquad\int_0^\eta\!\left(1-2\cos^2\eta^\prime\right)\sec^4\eta^\prime\sin\eta\cos\eta\,\mathrm d\eta^\prime = -\frac{4}{3}\sin^2\eta+\frac{1}{3}\tan^2\eta \\
  &\qquad\int_0^\eta\!\sin\eta^\prime\cos\eta^\prime\sec^4\eta^\prime\left(1-2\cos^2\eta\right)\,\mathrm d\eta^\prime = \frac{1}{2}\left(2\cos^2\eta+\sec^2\eta-3\right) \\
  &= \frac{2\pi a^4}{3}\left[\ln\sec^2\eta-\sec^2\eta+4-4\sin^2\eta+tan^2\eta-3\cos^2\eta\right] \\
  &= \frac{2\pi a^4}{3}\left[\ln\sec^2\eta-\sec^2\eta+\tan^2\eta+\cos^2\eta\right]
\end{align}

\begin{equation}
\rho = \frac{3\sec^4\eta}{\left(2+\sec^2\eta_0\right)\tan\eta_0}
\end{equation}

\begin{align}
\langle\bar{k_o}\rangle &= \delta\int_0^{\eta_0}\!\rho\left(\eta\right)V_p\left(\eta\right)\,\mathrm d\eta \\
  &= \frac{2\pi\delta a^4}{\left(2+\sec^2\eta_0\right)\tan\eta_0}\int_0^{\eta_0}\!\left[\sec^4\eta\ln\sec^2\eta - \sec^6\eta + \tan^2\eta\sec^4\eta+\sec^2\eta\right]\,\mathrm d\eta \\
  &\qquad\begin{aligned}
    \int_0^{\eta_0}\!\sec^4\eta\ln\sec^2\eta\,\mathrm d\eta = &\frac{4}{3}\eta_0 - \frac{10}{9}\tan\eta_0 + \frac{4}{3}\tan\eta_0\ln\sec\eta_0 \\
    &-\frac{2}{9}\sec^2\eta_0\tan\eta_0 + \frac{2}{3}\sec^2\eta_0\tan\eta_0\ln\sec\eta_0
  \end{aligned} \\
  &\qquad\int_0^{\eta_0}\!\sec^6\eta\,\mathrm d\eta = \frac{8}{15}\tan\eta_0 + \frac{4}{15}\tan\eta_0\sec^2\eta_0 + \frac{1}{5}\tan\eta_0\sec^4\eta_0 \\
  &\qquad\int_0^{\eta_0}\!\tan^2\eta\sec^4\eta\,\mathrm d\eta = -\frac{2}{15}\tan\eta_0 - \frac{1}{15}\tan\eta_0\sec^2\eta_0 + \frac{1}{5}\tan\eta_0\sec^4\eta_0 \\
  &\qquad\int_0^{\eta_0}\!\sec^2\eta\,\mathrm d\eta = \tan\eta_0 \\
  &=\frac{2\pi\delta a^4}{2+\sec^2\eta_0}\left[\frac{4}{3}\frac{\eta_0}{\tan\eta_0} - \frac{7}{9} + \frac{4}{3}\ln\sec\eta_0 - \frac{5}{9}\sec^2\eta_0 + \frac{2}{3}\sec^2\eta_0\ln\sec\eta_0\right]
\end{align}

\begin{equation}
\begin{split}
\langle\bar{k}\rangle &= 2\langle\bar{k_o}\rangle \\
  &= \frac{4}{9}\frac{\pi\delta a^4}{2+\sec^2\eta_0}\left[12\left(\frac{\eta_0}{\tan\eta_0}+\ln\sec\eta_0\right)+\left(6\ln\sec\eta_0-5\right)\sec^2\eta_0-7\right]
\end{split}
\end{equation}

\section{Expected Isolated Nodes}
\subsection{1+1 Dimensions}

Poisson point process:

\begin{equation}
P(x) = \frac{\left(\delta V\right)^x}{x!}e^{-\delta V}
\end{equation}

The probability a node is isolated at conformal time $\eta$ is given by this expression, where $x=0$ is the expected number of nodes in the sum of the light cone volumes $V=V_p\left(\eta\right)+V_f\left(\eta\right)$.  Manipulating this expression yields

\begin{equation}
\begin{split}
P\left(0\right) &= e^{-2\delta a^2\left[\left(\eta_0-\eta\right)\tan\eta_0 + \ln\sec^2\eta - \ln\sec\eta_0\right]} \\
  &= e^{-2\delta a^2\left[\eta_0\tan\eta_0 - \ln\sec\eta_0\right]}e^{-2\delta a^2\left[\ln\sec^2\eta - \eta\tan\eta_0\right]} \\
  &=\xi e^{-2\delta a^2\left[\ln\sec^2\eta-\eta\tan\eta_0\right]}
\end{split}
\end{equation}

Then, the expected number of isolated nodes is given by

\begin{equation}
\begin{split}
\langle N\left(0\right)\rangle &= N\int_0^{\eta_0}\!\rho\left(\eta\right)P\left(0\right)\,\mathrm d\eta \\
  &= \frac{N\xi}{\tan\eta_0}\int_0^{\eta_0}\!\sec^2\eta e^{-2\delta a^2\left[\ln\sec^2\eta - \eta\tan\eta_0\right]}\,\mathrm d\eta \\
  &\qquad e^{-2\delta a^2\ln\sec^2\eta} = \left(\cos\eta\right)^{4\delta a^2} \\
  &= \frac{N\xi}{\tan\eta_0}\int_0^{\eta_0}\left(\cos\eta\right)^{4\delta a^2 - 2} e^{\left(2\delta a^2 \tan\eta_0\right)\eta}\,\mathrm d\eta
\end{split}
\end{equation}

\end{document}
