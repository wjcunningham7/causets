\documentclass[12pt]{article}
\usepackage[top=1in, bottom=1in, left=1in, right=1in]{geometry}
\usepackage{mathtools}
\usepackage{amsmath}
\usepackage{amsfonts}
\usepackage{amssymb}
\DeclareMathSymbol{\comma}{\mathpunct}{letters}{"3B}
\begin{document}
\title{Causal Set Research Notes}
\author{Will Cunningham}
\maketitle

\section{Patch Volume}
\subsection{1+1 Dimensions}

\begin{equation}
\begin{split}
\mathrm{d}V &= (a\mathrm{sec}\eta)^{\mathrm{d}+1} \, \mathrm{d}\eta \, \mathrm{d}\Phi_d \\
   &= a^2\mathrm{sec}^2\eta \, \mathrm{d}\eta \, \mathrm{d}\theta
\end{split}
\end{equation}

\begin{equation}
\begin{split}
V &= a^2\int_0^{\eta_0} \! \mathrm{d}\eta \int_0^{2\pi} \! \mathrm{d}\theta \, \mathrm{sec}^2\eta \\
  &= 2\pi a^2 \int_0^{\eta_0} \! \mathrm{sec}^2\eta \, \mathrm{d}\eta \\
  &= 2\pi a^2\mathrm{tan}\eta_0
\end{split}
\end{equation}

\begin{equation}
\begin{split}
N &= \delta V \\
\therefore N_{\mathrm{1+1}} &= 2\pi\delta a^2 \mathrm{tan}\eta_0
\end{split}
\end{equation}

\subsection{3+1 Dimensions}

\begin{equation}
\mathrm{d}V = a^4 \mathrm{sec}^4\eta \, \mathrm{d}\eta \, \mathrm{sin}^2\phi \, \mathrm{sin}\chi \, \mathrm{d}\phi \, \mathrm{d}\chi \, \mathrm{d}\theta
\end{equation}

\begin{equation}
\begin{split}
V &= a^4 \int_0^{\eta_0} \! \mathrm{d}\eta \int_0^{2\pi} \! \mathrm{d}\theta \int_0^\pi \! \mathrm{d}\chi \int_0^\pi \! \mathrm{d}\phi \, \mathrm{sec}^4\eta \, \mathrm{sin}^2\phi \, \mathrm{sin}\chi \\
  &= 2\pi a^4 \int_0^{\eta_0} \! \mathrm{d}\eta \int_0^\pi \! \mathrm{d}\chi \int_0^\pi \! \mathrm{d}\phi \, \mathrm{sec}^4\eta \, \mathrm{sin}^2\phi \, \mathrm{sin}\chi \\
  &= 4\pi a^4 \int_0^{\eta_0} \! \mathrm{d}\eta \int_0^\pi \! \mathrm{d}\phi \, \mathrm{sec}^4\eta \, \mathrm{sin}^2\phi \\
  &= 2\pi^2 a^4 \int_0^{\eta_0} \! \mathrm{sec}^4\eta \, \mathrm{d}\eta \\
  &= \frac{2}{3}\pi^2 a^4 \, (2+\mathrm{sec}^2\eta_0) \, \mathrm{tan}\eta_0
\end{split}
\end{equation}

\begin{equation}
\begin{split}
N &= \delta V \\
\therefore N_{\mathrm{3+1}} &= \frac{2}{3}\pi^2\delta a^4 \, (2+\mathrm{sec}^2\eta_0) \, \mathrm{tan}\eta_0
\end{split}
\end{equation}

\section{Expected Average Degrees}
\subsection{1+1 Dimensions}

\begin{equation}
\rho = \frac{\mathrm{sec}^2\eta}{\mathrm{tan}\eta_0}
\end{equation}

\begin{equation}
V_p = \mathrm{ln} \, \mathrm{sec}\eta
\end{equation}

\begin{equation}
\begin{split}
\langle V_p \rangle &= \int_0^{\eta_0} \! \frac{\mathrm{sec}^2\eta}{\mathrm{tan}\eta_0} \, \mathrm{ln} \, \mathrm{sec}\eta \, \mathrm{d}\eta \\
  &= \frac{1}{\mathrm{tan}\eta_0} \, \lbrack \eta + \mathrm{tan}\eta (\mathrm{ln} \, \mathrm{sec}\eta - 1) \rbrack_0^{\eta_0} \\
  &= \frac{1}{\mathrm{tan}\eta_0} \, \lbrack \eta_0 + \mathrm{tan}\eta_0 (\mathrm{ln} \, \mathrm{sec}\eta_0 - 1) \rbrack \\
  &= \frac{\eta_0}{\mathrm{tan}\eta_0} + \mathrm{ln} \, \mathrm{sec}\eta_0 - 1
\end{split}
\end{equation}

\begin{equation}
V_f = (\eta_0 - \eta) \, \mathrm{tan}\eta_0 + \mathrm{ln} \, \left(\frac{\mathrm{sec}\eta}{\mathrm{sec}\eta_0}\right)
\end{equation}

\begin{equation}
\begin{split}
\langle V_f \rangle &= \int_0^{\eta_0} \! \frac{\mathrm{sec}^2\eta}{\mathrm{tan}\eta_0} \left[ (\eta_0 - \eta) \, \mathrm{tan}\eta_0 + \ln\left(\frac{\sec\eta}{\sec\eta_0}\right) \right] \, \mathrm{d}\eta \\
  &= \frac{1}{\tan\eta_0} \left[ \eta_0\tan\eta_0 \int_0^{\eta_0} \! \sec^2\eta \, \mathrm{d}\eta - \tan\eta_0 \int_0^{\eta_0} \! \eta\sec^2\eta \, \mathrm{d}\eta \right. \\ & \left.\,\,\,\,\, + \int_0^{\eta_0} \! \sec^2\eta\ln\sec\eta \, \mathrm{d}\eta - \ln\sec\eta_0 \int_0^{\eta_0} \! \sec^2\eta \, \mathrm{d}\eta \right] \\
  &= \frac{1}{\tan\eta_0} \left[ \eta_0\tan^2\eta_0 - \eta_0\tan^2\eta_0 + \tan\eta_0\ln\sec\eta_0 \right. \\ & \left. \,\,\,\,\, + \eta_0 + \tan\eta_0\left(\ln\sec\eta_0 - 1\right) - \tan\eta_0\ln\sec\eta_0 \right] \\
  &= \frac{\eta_0}{\tan\eta_0} + \ln\sec\eta_0 - 1
\end{split}
\end{equation}

\begin{equation}
\begin{split}
\therefore \langle V_p \rangle &= \langle V_f \rangle \\
\therefore \langle \bar{k} \rangle &= 2a^2\delta \left( \langle V_p \rangle + \langle V_f \rangle \right) \\
  &= 4 \delta a^2 \left( \frac{\eta_0}{\tan\eta_0} + \ln\sec\eta_0 - 1 \right)
\end{split}
\end{equation}

\subsection{3+1 Dimensions}

\begin{equation}
\rho = \frac{3\sec^4\eta}{\left( 2 + \sec^2\eta_0 \right) \tan\eta_0}
\end{equation}

\begin{equation}
\begin{split}
V_p(\eta) &= \int_0^\eta \! \mathrm d \eta^\prime \int_0^{\eta - \eta^\prime} \! 2\mathrm d \theta \int_0^\pi \! \mathrm d \phi \int_0^\pi \! \mathrm d \chi \, a^4\sec^4\eta^\prime \, \sin^2\phi \, \sin\chi \\
  & \;\;\;\;\;\;\;\;\;\; \int_0^\pi \! \sin\chi \, \mathrm d \chi = 2 \\
  & \;\;\;\;\;\;\;\;\;\; \int_0^\pi \! \sin^2\phi \, \mathrm d \phi = \frac{\pi}{2} \\
  & \;\;\;\;\;\;\;\;\;\; \int_0^{\eta - \eta^\prime} \! 2\,\mathrm d\theta = 2\left(\eta - \eta^\prime\right) \\
  &= 2\pi a^4 \int_0^\eta \! \left(\eta - \eta^\prime\right)\sec^4\eta^\prime\,\mathrm d\eta^\prime \\
  & \;\;\;\;\;\;\;\;\;\; \int_0^\eta \! \sec^4\eta^\prime \, \mathrm d\eta^\prime = \frac{1}{3}\tan\eta\left(2+\sec^2\eta\right) \\
  & \;\;\;\;\;\;\;\;\;\; \int_0^\eta \! \eta^\prime\sec^4\eta^\prime\,\mathrm d\eta^\prime = \frac{1}{3}\left(-2\ln\sec\eta - \frac{1}{2}\sec^2\eta + 2\eta\tan\eta + \eta\sec^2\eta\tan\eta + \frac{1}{2}\right) \\
  &= 2\pi a^4\left(\frac{2}{3}\eta\tan\eta + \frac{1}{3}\eta\sec^2\eta + \frac{2}{3}\ln\sec\eta + \frac{1}{6}\sec^2\eta - \frac{2}{3}\eta\tan\eta - \frac{1}{3}\eta\sec^2\eta\tan\eta - \frac{1}{6}\right) \\
  &= 2\pi a^4\left(\frac{1}{3}\eta\sec^2\eta + \frac{2}{3}\ln\sec\eta + \frac{1}{6}\sec^2\eta - \frac{1}{3}\eta\sec^2\eta\tan\eta - \frac{1}{6}\right) \\
  &= \frac{2}{3}\pi a^4 \left(\eta\sec^2\eta + 2\ln\sec\eta + \frac{1}{2}\sec^2\eta - \eta\sec^2\eta\tan\eta - \frac{1}{2}\right)
\end{split}
\end{equation}

\begin{align}
\langle\bar{k_o}\rangle &= \int_0^{\eta0} \! \rho\delta V_p(\eta)\,\mathrm d\eta \\
  &\begin{aligned}
    = \frac{2\pi\delta a^4}{(2+\sec^2\eta_0)\tan\eta_0} \int_0^{\eta_0} &\left(\eta\sec^6\eta + \sec^4\ln\sec^2\eta + \frac{1}{2}\sec^6\eta \right. \\ 
    & \quad \left. - \eta\sec^6\eta\tan\eta - \frac{1}{2}\sec^4\eta\right)\,\mathrm d\eta
  \end{aligned} \\
  &\qquad\begin{aligned}
    \int_0^{\eta_0}\! \eta\sec^6\eta\,\mathrm d\eta = &-\frac{8}{15}\ln\sec\eta_0 - \frac{2}{15}\sec^2\eta_0 - \frac{1}{20}\sec^4\eta_0 + \frac{8}{15}\eta_0\tan\eta_0 \\ 
    & + \frac{4}{15}\eta_0\sec^2\eta_0\tan\eta_0 + \frac{1}{5}\eta_0\sec^4\eta_0\tan\eta_0 + \frac{11}{60}
  \end{aligned} \\
  &\qquad\begin{aligned}
    \int_0^{\eta_0} \! \sec^4\eta\ln\sec\eta\,\mathrm d\eta = &\frac{2}{3}\eta_0 - \frac{5}{9}\tan\eta_0 - \frac{1}{9}\sec^2\eta_0\tan\eta_0 \\
    & + \frac{1}{3}\ln\sec\eta_0\sec^2\eta_0\tan\eta_0 + \frac{2}{3}\ln\sec\eta_0\tan\eta_0
  \end{aligned} \\
  &\qquad\begin{aligned}
    \int_0^{\eta_0} \! \sec^6\eta\,\mathrm d\eta = &\frac{8}{15}\tan\eta_0 + \frac{4}{15}\sec^2\eta_0\tan\eta_0 \\ 
    & + \frac{1}{5}\sec^4\eta_0\tan\eta_0
  \end{aligned} \\
  &\qquad\begin{aligned}
    \int_0^{\eta_0} \! \eta\sec^6\eta\tan\eta\,\mathrm d\eta = &\frac{1}{6}\eta_0\sec^6\eta_0 - \frac{4}{45}\tan\eta_0 \\ 
    & - \frac{2}{45}\sec^2\eta_0\tan\eta_0 - \frac{1}{30}\sec^4\eta_0\tan\eta_0
  \end{aligned} \\
  &\qquad\int_0^{\eta_0} \! \sec^4\eta\,\mathrm d\eta = \frac{2}{3}\tan\eta_0 + \frac{1}{3}\sec^2\eta_0\tan\eta_0 \\
  &= \frac{2\pi\delta a^4}{\left(2+\sec^2\eta_0\right)\tan\eta_0}I
\end{align}

\begin{align}
I = &-\frac{8}{15}\ln\sec\eta_0 - \frac{2}{15}\sec^2\eta_0 - \frac{1}{20}\sec^4\eta_0 + \frac{8}{15}\eta_0\tan\eta_0 + \frac{4}{15}\eta_0\sec^2\eta_0\tan\eta_0 \notag \\
  & + \frac{1}{5}\eta_0\sec^4\eta_0\tan\eta_0 + \frac{11}{60} + \frac{4}{3}\eta_0 - \frac{49}{45}\tan\eta_0 - \frac{19}{90}\sec^2\eta_0\tan\eta_0 \notag \\
  & + \frac{2}{3}\ln\sec\eta_0\sec^2\eta_0\tan\eta_0 + \frac{4}{3}\ln\sec\eta_0\tan\eta_0 + \frac{2}{15}\sec^4\eta_0\tan\eta_0 \notag \\
  &- {1}{6}\eta_0\sec^6\eta_0
\end{align}

\begin{align}
\langle\bar{k}\rangle &= 2\langle\bar{k_o}\rangle \\
  &\begin{aligned}
    = &\frac{4}{9}\frac{\pi\delta a^4}{2+\sec^2\eta_0} \left[12\left(\eta_0\cot\eta_0 + \ln\sec\eta_0\right) + \left(6\ln\sec\eta_0 -\frac{19}{10}\right) \sec^2\eta_0 \right. \\
    & \left. - \frac{49}{5} - \frac{24}{5}\ln\sec\eta_0\cot\eta_0 - \frac{6}{5}\sec^2\eta_0\cot\eta_0 - \frac{9}{20}\sec^4\eta_0\cot\eta_0 \right. \\
    & \left. + \frac{24}{5}\eta_0 + \frac{12}{5}\eta_0\sec^2\eta_0 + \frac{9}{5}\eta_0\sec^4\eta_0 + \frac{33}{20}\cot\eta_0 + \frac{6}{5}\sec^4\eta_0 \right. \\
    & \left. - \frac{3}{2}\eta_0\sec^6\eta_0\cot\eta_0\right]
  \end{aligned}
\end{align}

\section{Expected Isolated Nodes}
\subsection{1+1 Dimensions}

The discrete Poisson distribution is given by

\begin{equation}
P(X=x) = \frac{\left(\delta V\right)^x}{x!}e^{-\delta V}
\end{equation}

which describes the probability $X$ points fall in the volume $V$.  The respective volumes of the past and future light cones at conformal time $\eta$ are given by

\begin{equation}
\begin{split}
V_p\left(\eta\right) &= 2a^2\ln\sec\eta \\
V_f\left(\eta\right) &= 2a^2\left[\left(\eta_0-\eta\right)\tan\eta_0 + \ln\left(\frac{\sec\eta}{\sec\eta_0}\right)\right] \\
V &= 2\pi a^2 \tan\eta_0
\end{split}
\end{equation}

where $V$ is the total volume of the spacetime patch on the de Sitter manifold.  The volume of the region outside a given node's past and future light cones is described by

\begin{align}
\phantom{\delta V_0\left(\eta\right)}
V_o\left(\eta\right) &= V - \left(V_p\left(\eta\right)+V_f\left(\eta\right)\right) \\
  &\begin{aligned}
    = 2a^2 & \left[\pi\tan\eta_0 - 2\ln\sec\eta + \left(\eta-\eta_0\right)\tan\eta_0 \right. \\
    & \quad \left. + \ln\sec\eta_0 \right]
  \end{aligned} \\
  &\begin{aligned}
    = 2a^2 & \left[\left(\eta\tan\eta_0 - \ln\sec^2\eta\right) + \left(\pi\tan\eta_0 \right. \right. \\
    & \quad \left. \left. -\eta_0\tan\eta_0 + \ln\sec\eta_0\right)\right]
  \end{aligned} \\
  &= 2a^2 \left(\mu\left(\eta\right) + \xi\right) \\
\delta V_o\left(\eta\right) &= 2a^2\delta\left(\mu\left(\eta\right) + \xi\right)
\end{align}

where we have defined the functions

\begin{equation}
\begin{split}
\mu\left(\eta\right) &\equiv \eta\tan\eta_0 - \ln\sec^2\eta \\
\xi & \equiv \pi\tan\eta_0 - \eta_0\tan\eta_0 + \ln\sec\eta_0
\end{split}
\end{equation}

Thus, the continuous Poissonian distribution is

\begin{equation}
P\left(N\comma \eta\right) = \frac{\left(2a^2\delta\right)^N}{\Gamma\left(N+1\right)}\left(\mu\left(\eta\right) + \xi\right)^N e^{-2a^2\delta\left(\mu\left(\eta\right) + \xi\right)}
\end{equation}

which describes the probability $N$ points lie outside the light cones at conformal time $\eta$.  Therefore, the expected number of isolated nodes is simply

\begin{equation}
\langle N\left(0\right)\rangle = N\int_0^{\eta_0} \! \rho\left(\eta\right)P\left(N-1\comma\eta\right) \, \mathrm d\eta
\end{equation}

To evaluate the integral, we must expand the term $\left(\mu\left(\eta\right)+\xi\right)^N$ using a Binomial expansion.  This is allowed if $N$ is large enough.  Note that we use $N-1$ because the point whose light cones we are considering is not included in the Poisson point process.  To do this trick, consider $\alpha \equiv \frac{\pi}{2} - \eta$.  
\begin{equation}
\begin{split}
\left(\mu\left(\eta\right) + \xi\right)^N &= x - m\left(\alpha\right) \\
  &= x\left(1-\frac{m\left(\alpha\right)}{x}\right)
\end{split}
\end{equation}

where we have defined the new quantities

\begin{equation}
\begin{split}
m & \equiv \alpha\tan\eta_0 + \ln\csc^2\alpha \\
x & \equiv \frac{3\pi}{2}\tan\eta_0 - \eta_0\tan\eta_0 + \ln\sec\eta_0
\end{split}
\end{equation}

Now, we may use the binomial expansion:

\begin{equation}
\begin{split}
\left(1-\frac{m\left(\alpha\right)}{x}\right)^N &\approx 1 - \frac{N}{x}m\left(\alpha\right) + \frac{N\left(N-1\right)}{2}\left(\frac{m\left(\alpha\right)}{x}\right)^2 - \frac{N\left(N-1\right)\left(N-2\right)}{6}\left(\frac{m\left(\alpha\right)}{x}\right)^3 \\
  & \qquad m\left(\alpha\right) = \left(\frac{\pi}{2} - \eta\right)\tan\eta_0 + \ln\sec^2\eta \\
  & \approx 1 - \frac{N}{x}\left[\left(\frac{\pi}{2} - \eta\right)\tan\eta_0 + \ln\sec^2\eta\right]
\end{split}
\end{equation}

and use the following approximations:

\begin{equation}
\begin{split}
m &\approx \frac{\pi}{2}\tan\eta_0 - \eta\tan\eta_0 \\
m^2 &\approx \pi\tan\eta_0\left(\frac{\pi}{4}\tan\eta_0 - \eta_0\right) + \eta^2\tan^2\eta_0 \\
x &\approx \left(\frac{3\pi}{2}-\eta_0\right)\tan\eta_0
\end{split}
\end{equation}

so that the distribution is now

\begin{align}
P\left(N\comma\eta\right)
  &\begin{aligned}
     = \frac{\left[2a^2\delta\left(\frac{3\pi}{2}-\eta_0\right)\tan\eta_0\right]^N}{\Gamma\left(N+1\right)}e^{-2a^2\delta\left(\pi-\eta_0\right)\tan\eta_0}
  \end{aligned} \notag \\
  &\quad\begin{aligned}
    \left[1-N\left(\frac{\pi-2\eta}{3\pi-2\eta_0}\right) + \frac{N\left(N-1\right)}{2}\frac{\pi\left(\frac{\pi}{4}-\eta_0\cot\eta_0\right) + \eta^2}{\left(\frac{3\pi}{2}-\eta_0\right)^2}\right]e^{-2a^2\delta\eta\tan\eta_0}
  \end{aligned} \\
  &\begin{aligned}
    = \Xi\left(N\right)\left[1-N\left(\frac{\pi-2\eta}{3\pi-2\eta_0}\right) + \frac{N\left(N-1\right)}{2}\frac{\pi\left(\frac{\pi}{4}-\eta_0\cot\eta_0\right) + \eta^2}{\left(\frac{3\pi}{2}-\eta_0\right)^2}\right]e^{-2a^2\delta\eta\tan\eta_0}
  \end{aligned} \\
\langle N\left(0\right)\rangle
  &\begin{aligned}
    = \frac{N}{\tan\eta_0}\Xi\left(N\right) \int_0^{\eta_0} \! \sec^2\eta e^{-2a^2\delta\eta\tan\eta_0}
  \end{aligned} \notag \\
  &\quad\begin{aligned}
    \left[1-N\left(\frac{\pi-2\eta}{3\pi-2\eta_0}\right) + \frac{N\left(N-1\right)}{2}\frac{\pi\left(\frac{\pi}{4}-\eta_0\cot\eta_0\right) + \eta^2}{\left(\frac{3\pi}{2}-\eta_0\right)^2}\right] \, \mathrm d\eta
  \end{aligned}
\end{align}

where the function $\Xi\left(N\right)$ has been defined to be

\begin{equation}
\Xi\left(N\right) \equiv \frac{\left[2a^2\delta\left(\frac{3\pi}{2}-\eta_0\right)\tan\eta_0\right]^N}{\Gamma\left(N+1\right)}e^{-2a^2\delta\left(\pi-\eta_0\right)\tan\eta_0}
\end{equation}

\end{document}
